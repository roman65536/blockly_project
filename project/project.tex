\documentclass[a4paper,12pt]{report}



\usepackage[utf8]{luainputenc}
\usepackage[german]{babel}
\usepackage[textwidth=483.65pt,textheight=700.80pt]{geometry}
\usepackage{enumitem}
%\usepackage{rotating}
\usepackage{graphicx}
\usepackage{fancyhdr}
%\usepackage{scrpage2}
\usepackage{longtable}
\usepackage{verbatim}
\usepackage{varioref}
\usepackage{hyperref}
\usepackage{multicol}
\usepackage{multirow}
\usepackage{color}
\usepackage{array}
\usepackage{colortbl}
%\usepackage{lscape}
\usepackage{pdflscape}
\usepackage{totpages}
\usepackage[table]{xcolor}
%\usepackage{piechart}
%\usepackage{dotchart}
%\usepackage{nofloat}
\usepackage{pgfgantt}
\usepackage{luacode}

\usepackage[rigidchapters,explicit]{titlesec}

%%\directlua{package.cpath = "/usr/local/lib/lua/5.3/luasql/postgres.so"}
%%\directlua{require("luasql.mysql")}

\begin{luacode*}
  package.cpath = "/usr/local/lib/lua/5.3/luasql/postgres.so"  
local driver = require "luasql.postgres"
-- create environment object
env = assert (driver.postgres())
-- connect to data source
con = assert (env:connect("host=127.0.0.1 port=5432 user=roman password=roman dbname=project"))
cur = assert (con:execute"SELECT *  FROM pg_catalog.pg_tables ;")
col=cur:getcolnames()
col_name=""
for i = 1, #col do
 col_name=col_name..string.format("%s ",col[i])
end
print(col_name)
row = cur:fetch ({}, "n")
while row do
row_val=""
  for i =1, #row do
	   if row[i] ~= nil then
             row_val=row_val .. string.format("%s ",row[i] )
	   else
             row_val=row_val .. "- " 
	   end
   end
	   print(row_val)
  --print(string.format(" %s,  %s", row[1], row[2]))
  -- reusing the table of results
  row = cur:fetch (row, "n")
end


project_id=arg[3]

   local query="select name,p_id,descr from project where p_id="..project_id..";"
   --tex.sprint("\\bold{"..query.."}\n")
   print(query)
        local cur_c= con:execute(query)
        local   row1 =cur_c:fetch ({}, "a")
	prj_name=row1.name
	cust_name=row1.name



\end{luacode*}

\xdefinecolor{TableLightGray}{rgb}{0.90,0.90,0.90}

\hypersetup{
pdftitle={Project}
pdfauthor={Pollak's Software}
}
\pdfcompresslevel 9
\pagestyle{fancyplain}
\rhead{}
\cfoot{\thepage{}/\ref{TotPages}}

\definecolor{td2}{rgb}{.93,.54,.11}
\definecolor{td}{rgb}{.22,.28,.31}

\makeatletter
\def\thickhrulefill{\leavevmode \leaders \hrule height 1pt\hfill \kern \z@}
\def\maketitle{%
 \null
\thispagestyle{empty}
 \setlength{\unitlength}{1mm}
\begin{picture}(0,0)
%%  \put(-5,-5){\color{white}\includegraphics[width=2.5in]{TDC_logo}}
%%  \put(-5,-5){\color{white}\includegraphics[width=2.5in]{TDC_logo}}
%%   \put(-5,-5){\color{white}\includegraphics[width=3.5in]{TER_Orange_Logo} \includegraphics[width=2.5in]{TER_4C_NOTAG}}
   \put(-5,-239){\color{black}\@cnt_url}
 \end{picture}

 \begin{minipage}{0.99\linewidth}
 \begin{minipage}{0.25\linewidth}
  \hfill
  \vfill
 \end{minipage}
 \begin{minipage}{0.75\linewidth}
  \vskip 5cm
  \color{td}\Huge{\textbf{\textsf{Projekt\newline{\@customer}\newline\today}}}
   \vskip 6.5cm
%%  \includegraphics[scale=.8]{\@cust_logo}\\
%%  \color{black}\Huge{\textbf{\textsf{\@customer}}}
\end{minipage}
 \end{minipage}
}
\def\customer#1{\def\@customer{#1}}
\def\period#1{\def\@period{#1}}
\def\cnt_url#1{\def\@cnt_url{#1}}
\def\cust_logo#1{\def\@cust_logo{#1}}

\begin{document}

\renewcommand{\footrulewidth}{0.4pt}
\renewcommand \thesection {\@arabic\c@section}
\renewcommand{\arraystretch}{1.5}

%%\customer{T-Shirt}
\customer{\directlua{tex.print(prj_name)}}
\period{\today}
\cnt_url{www.yourcompany.ch}
\noindent
\maketitle
\newpage
%\begin{bf}THE INFORMATION CONTAINED IN THIS DOCUMENT IS PROVIDED ON AN “AS-IS” BASIS, WITHOUT WARRANTY OF ANY KIND, EITHER
EXPRESS OR IMPLIED, INCLUDING THE IMPLIED WARRANTIES OF MERCHANTABILITY, FITNESS FOR A PARTICULAR PURPOSE, OR
NON-INFRINGEMENT. SOME JURISDICTIONS DO NOT ALLOW THE EXCLUSION OF IMPLIED WARRANTIES, SO THE ABOVE EXCLUSION
MAY NOT APPLY TO YOU. IN NO EVENT WILL TERADATA CORPORATION BE LIABLE FOR ANY INDIRECT, DIRECT, SPECIAL, INCIDENTAL,
OR CONSEQUENTIAL DAMAGES, INCLUDING LOST PROFITS OR LOST SAVINGS, EVEN IF EXPRESSLY ADVISED OF THE POSSIBILITY OF
SUCH DAMAGES.\\
\end{bf}


\begin{small}
The information contained in this document may contain references or cross-references to features, functions, products, or services that are
not announced or available in your country. Such references do not imply that Teradata Corporation intends to announce such features,
functions, products, or services in your country. Please consult your local Teradata Corporation representative for those features, functions,
products, or services available in your country.\\
Information contained in this document may contain technical inaccuracies or typographical errors. Information may be changed or updated
without notice. Teradata Corporation may also make improvements or changes in the products or services described in this information at any
time without notice.\\
To maintain the quality of our products and services, we would like your comments on the accuracy, clarity, organization, and value of this
document. Please e-mail: teradata-books@lists.teradata.com\\
Any comments or materials (collectively referred to as “Feedback”) sent to Teradata Corporation will be deemed non\-confidential. Teradata
Corporation will have no obligation of any kind with respect to Feedback and will be free to use, reproduce, disclose, exhibit, display, transform,
create derivative works of, and distribute the Feedback and derivative works thereof without limitation on a royalty-free basis. Further, Teradata
Corporation will be free to use any ideas, concepts, know-how, or techniques contained in such Feedback for any purpose whatsoever, including
developing, manufacturing, or marketing products or services incorporating Feedback.\\
\bf{Copyright © 2005 - 2013 by Teradata Corporation. All Rights Reserved.}
\end{small}

\tableofcontents


%\titleformat{\section}
%{}{}{0em}{\colorbox{td}{\parbox{17cm}{\normalfont\large\bfseries\color{white}\thesection\hspace{.3cm} #1}}}


\setcounter{chapter}{0}


\begin{luacode*}



function get_party(prj, st)

local query = "select party_id,p_func from party where p_id="..prj.." and s_id =" ..st..";"
print(query)
tex.sprint("\\begin{itemize}")
 local cur_c= con:execute(query)
 local   row1 =cur_c:fetch ({}, "a")
 
 while row1 do
     tex.sprint("\\item{".. row1.p_func .. ":} ")
     print(string.format("%s ",row1['p_func']))
     local query1 = "select p_func,p_fname,p_name,p_email from party_memb where p_id="..prj.." and party_id =" ..row1['party_id'] ..";"
     print(query1)
    local cur_c1= con:execute(query1)
    local   row2 =cur_c1:fetch ({}, "a")

    tex.sprint("\\begin{longtable}{p{2cm}p{2cm}p{2cm}p{5cm}}")
    
    while row2 do
     tex.sprint(string.format("%s & %s & %s & %s\\\\",row2['p_func'],row2['p_fname'],row2['p_name'],row2['p_email']))
    row2 =cur_c1:fetch (row2, "a")
    end
     tex.sprint("\\end{longtable}")
   
 row1 =cur_c:fetch (row1, "a")
 end
 tex.sprint("\\end{itemize}")
end


function conv_m(text)

 text=string.gsub(text,"&#39","\\verb|'|")
 text=string.gsub(text,"&#40","(")
 text=string.gsub(text,"&#41",")")
 text=string.gsub(text,"&","\\verb|&|")
 text=string.gsub(text,"#","\\verb|#|")
 text=string.gsub(text,"_","\\textunderscore{}")
 --text=string.gsub(text,"^","\\verb|^|")
 text=string.gsub(text,"/","/\\allowbreak{}")
 --text=string.gsub(text,"%c","")
 test=string.gsub(text,"\026","")

return text
end



function get_doc(nr)

return nr
end


mcr={}
mcr["PRJ"]=cust_name

tbl={}
tbl["MIL"]="Meilenstein"
tbl["ADD_PH"]="Projekt Phase"
tbl["ADD"]="Zusätzlich"


--[[
phs={}
phs[1]="Inittiation"
phs[2]="Planning"
phs[3]="Executing"
phs[4]="Monitoring and Controlling"
phs[5]="Closing"

--]]


phs={}
phs[1]="Initialisierung"
phs[2]="Plannug"
phs[3]="Ausführung"
phs[4]="Überwachung und Kontrolle"
phs[5]="Ende"
phs[6]="Mission Analysis"
phs[7]="Mission Definition Review (MDR)"
phs[8]="Feasibility/Conception"
phs[9]="Preliminary Requirement Review (PRR)"
phs[10]="Preliminary Definition phase"
phs[11]="System Definition Review (SDR)"
phs[12]="Preliminary Design"
phs[13]="Preliminary Design Review (PDR)"
phs[14]="Final Design"
phs[15]="Critical Design Review (CDR)"
phs[16]="Qualification Review (QR)"
phs[17]="Acceptance Review (AR)"
phs[18]="Utilization"
phs[19]="Operational Readiness Review (ORR)"
phs[20]="Flight Readiness Review (FRR)"
phs[21]="Launch Readiness Review (LRR)"
phs[22]="Flight Qualification Review (FQR)"
phs[23]="End of Life Review (EOLR)"
phs[24]="Disposal"



function conv_macro(str)
str1=string.gsub(str,"$$(%w*)", function(a)  print(a) if (mcr[a] ~= nil ) then print (mcr[a] ) return mcr[a] end end)

return str1
end


project_id=arg[3]

   local query="select name,p_id,descr from project where p_id="..project_id..";"
   --tex.sprint("\\bold{"..query.."}\n")
   print(query)
        local cur_c= con:execute(query)
        local   row1 =cur_c:fetch ({}, "a")
	prj_name=row1.name
	cust_name=row1.name

   

    tex.sprint("\\chapter*{Projekt ``"..conv_m(prj_name).."'' }")
    tex.sprint(conv_m(row1.descr))
    tex.sprint("\\section{Projekt Teilnehmer}")
    get_party(project_id,0)
    tex.sprint("\\section{Projekt Schritte}")


    
    
               tex.sprint("\\begin{enumerate}")





		local cur2= con:execute("Select s_id, st_name,name,to_char(tt,'DD-MM-YYYY') as tt,cnt, st_add from steps where p_id="..project_id.." order by ord;")
                 local row2 =cur2:fetch ({}, "a")
		while row2 do
                        --tex.sprint(string.format("\\item{%s }\\newline",conv_m(tbl[row2.st_name]))) 
			mcr["SID"]=row2.name
                         
			mcr["DATE"]=row2.tt
			mcr["NR"]=row2.cnt
                        
		
			if (row2.st_name == "MIL" )  then
				mcr["NAME"]=row2.name
				str=get_doc(7)
				print(conv_macro(str)) 
              --  tex.sprint("\\item{"..conv_macro(str).." ".. row2.tt.. "}")
                tex.sprint("\\item{Meilenstein - "..row2.tt.." "..row2.name.. "}")
				
			elseif (row2.st_name == "ADD" )  then
				mcr["NAME"]=row2.name
				if (row2.st_add == "inc") then mcr["TYPE"]="Incident" else mcr["TYPE"]="Change Control" end
				str=get_doc(8)
				print(conv_macro(str)) 
                                tex.sprint("\\item{Task - "..row2.tt.." "..conv_macro(row2.name).."}")
			elseif (row2.st_name == "ADD_PH" )  then
			    mcr["NAME"]=row2.name
                print("add "..row2.st_add)
                print(phs[tonumber(row2.st_add)])
				if (row2.st_add == "inc") then mcr["TYPE"]="Incident" else mcr["TYPE"]="Change Control" end
				str=get_doc(9)
				print(conv_macro(str)) 
               -- tex.sprint("\\item{"..conv_macro(str).."}")
                tex.sprint("\\item{Projekt Phase "..phs[tonumber(row2.st_add)].. " - " ..row2.tt.. " ".. row2.name.. "}")

           	elseif (row2.st_name == "ADD_MET" )  then
                tex.sprint("\\item{Meeting - ".. row2.tt .." " .. row2.name .. "}\\\\")
                --tex.sprint(conv_m(row2.st_add))
                tex.sprint("Teilnehmer:\\\\")
			    get_party(project_id,row2.s_id)
                tex.sprint(conv_m(row2.st_add))

			end
				
                        print(row2.st_name.." & "..row2.name.."\\\\")
			print(tbl[row2.st_name])
                        row2 =cur2:fetch (row2, "a")
                 end
		tex.sprint("\\end{enumerate}")






\end{luacode*}

\end{document} 
